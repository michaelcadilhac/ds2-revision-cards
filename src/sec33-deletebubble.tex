\documentclass[12pt,letterpaper,USenglish]{article}

\usepackage[most]{tcolorbox}
\usepackage{amsmath}
\usepackage{amssymb}
\usepackage{microtype}
\usepackage{tikz}
\usepackage{paralist}
\usetikzlibrary{calc,positioning,backgrounds,intersections,graphdrawing,shapes,graphs,arrows,arrows.meta}
\usegdlibrary{trees}

%%% Please use define.org for macros.
\input{define.orgtex}

\begin{document}

\tikzset{lead/.style={anchor=north west,font=\it\bfseries,text=algs4red},
  treenode/.style={draw, minimum
      height=13pt, fill=white,inner sep=3pt, font=\footnotesize,
      rounded rectangle},
    algs4arrow/.style={->,>={Latex[length=7pt]},line width=0.8mm,algs4red},
    graphs/mytree/.style= {
    tree layout,
    sibling distance=5mm, level distance=5mm,
    nodes={treenode},
    edges={thick}}}

\lstset{columns=fullflexible}

\newsavebox{\codebox}% For saving code
\begin{lrbox}{\codebox}
\begin{lstalgs4}
|delete|(tree, key)$\rule[-10pt]{0cm}{0pt}$
    if key! is at the root of !tree
      if tree! is a leaf!
        !remove !key! from it and !return
      else
        subtree = !child after !key
        !replace !key! with !min(subtree)! in !tree!'s root!
        key = min(subtree)
    else
      subtree = !child in which !key! could be!
    |delete|(subtree, key)
    if subtree!'s root is a !1-!node!
      |bubbleup|(tree, subtree)
\end{lstalgs4}
\end{lrbox}


\newsavebox{\bubblebox}% For saving code
\begin{lrbox}{\bubblebox}
\begin{lstalgs4}
|bubbleup|(tree, subtree)$\rule[-10pt]{0cm}{0pt}$
    if subtree!'s left sibling is a !3-!node!
      !rebalance from left sibling!
    else if subtree!'s right sibling is a !3-!node!
      !rebalance from right sibling!
    else if subtree! has a left sibling!
      !merge left sibling with its parent's key!
    else
      !merge right sibling with its parent's key!
\end{lstalgs4}
\end{lrbox}

\noindent\begin{tikzpicture}[every node/.style={anchor=north west}]
  % Main background
  \begin{scope}[on background layer]
    \fill[black!10!white, draw=white, line width=6pt] (-.5\textwidth, 0) rectangle (.5\textwidth, -18cm);
  \end{scope}

  % Clip and left margin
  \path [save path=\theframe, rounded corners] ($(-.5\textwidth, 0) + (1.5pt,-1.5pt)$)
  rectangle ($(.5\textwidth, -18cm) + (-1.5pt, 1.5pt)$);
  \clip [use path=\theframe];
  \coordinate (left) at ($(-.5\textwidth, 0) + (0.3cm,0)$);

  % Title
  \node [anchor=north,text width=20cm,minimum height=1.4cm,align=center,fill=algs4red,text=white]
  {\sffamily\Large \textbf{3.3}\quad BALANCED SEARCH TREES: BUBBLE DELETION IN 2-3 TREES};

  % Main algorithm
  \begin{scope}[anchor=north west, shift={($(left) + (0.75, -2cm)$)}]
    % Background & lead
    \begin{scope}[on background layer]
      \fill[black!15] (-.5, 0) rectangle (18, -8.5);
    \end{scope}
    \node[lead,draw=none,shape=rectangle] (alg) at (-.3, -0.2) {Main algorithms.};
    

    \node [xshift=.5cm, yshift=-0cm, text width=15cm,anchor=north west] (code) at
    (alg.south west) {\usebox{\codebox}};
    \draw [thick]
    ($(code.north west) + (0.1, -0.7)$) -- +(5cm, 0) -|
    ($(code.north west) + (0.43, -0.9)$) |- ($(code.south west) + (1.8, -0.1)$);

    \node [xshift=9.5cm, yshift=-0cm, text width=15cm,anchor=north west] (code)
    at (alg.south west) {\usebox{\bubblebox}};
    \draw [thick]
    ($(code.north west) + (0.1, -0.7)$) -- +(5cm, 0) -|
    ($(code.north west) + (0.43, -0.9)$) |- ($(code.south west) + (1.8, -0.1)$);

    \node[yshift=-.2cm, font=\itshape,text width=8cm] at (code.south west)
    {\begin{compactitem}
      \item[\large\(\color{algs4red}+\)] After all the recursive calls to
        \lstinline{|delete|}, if \lstinline{tree}'s root is a \texttt{1-}node,
        then set \lstinline{tree} to be \lstinline{tree}'s only child (if
        non-\lstinline{null}).
      \end{compactitem}};

  \end{scope}

  
      
  % Rebalance
  \begin{scope}[shift={($(left) + (0.5cm,-11.4cm)$)},
    every node/.style={anchor=south},
    note/.style={font=\footnotesize},
    subtree/.style={draw=none,fill=none},
    edgenote/.style={algs4red,->,>=stealth',shorten >=3pt}]

    % Background & lead
    \begin{scope}[on background layer]
      \fill[black!15] (-0.07, 0.4) rectangle (9.5, -6);
    \end{scope}
    \node[lead,draw=none,shape=rectangle] at (0, 0.3) {Rebalancing.};

    %% Left heavy
    \graph [mytree] {
      root [x = 3cm,y=-1cm,draw=none,as={\phantom{\(2~~~7~~~9\)}}];
      root --[gray] { a [as={\color{gray}\(T_1\)},subtree]};
      root --
      { 46 [as={\(4~~6\)}] --
        { b [as={\(T_2\)},subtree],
          c [as={\(T_3\)},subtree],
          d [as={\(T_4\)},subtree] },
        es [as={},minimum size=2pt,fill=black] --
        { e [as={\(T_5\)},subtree]}};
      root --[gray]
      { f [as={\color{gray}\(T_6\)},subtree]}};
    \node[anchor=west, treenode,draw=gray] at (root.west) {\color{gray}\(2~~~7\)};
    \node[anchor=east, treenode,draw=gray,fill=none] at (root.east) {\(7~~~\color{gray}9\)};

    \node[font=\footnotesize] (t) at (4, -0.2) {\texttt{tree}};
    \draw (t.west) edge [bend right,edgenote] (root.north);

    \node[font=\footnotesize] (t) at (2.5, -3) {\texttt{subtree}};
    \draw[edgenote] (t.east) .. controls +(0.5cm, 0) and +(-0.5cm, -0.2cm) ..  (es.south west);

    \node[font=\itshape\footnotesize,text width=3cm, align=center] (g) at (7, -1)
    {\color{gray}Grayed parts may or may not be there};
    \draw (g.west) edge[bend right=20,edgenote,draw=gray] +(-1.3cm, -.5cm);

    %% Right heavy
    \graph [mytree] {
      root [x = 3cm,y=-4cm,draw=none,as={\phantom{\(2~~~4~~~9\)}}];
      root --[gray] { a [as={\color{gray}\(T_1\)},subtree]};
      root --
      { es [as={},minimum size=2pt,fill=black] --
        { e [as={\(T_2\)},subtree]},
       67 [as={\(6~~7\)}] --
        { b [as={\(T_3\)},subtree],
          c [as={\(T_4\)},subtree],
          d [as={\(T_5\)},subtree] }};
      root --[gray]
      { f [as={\color{gray}\(T_6\)},subtree]}};
    \node[anchor=west, treenode,draw=gray] at (root.west) {\color{gray}\(2~~~4\)};
    \node[anchor=east, treenode,draw=gray,fill=none] at (root.east) {\(4~~~\color{gray}9\)};

    %% Rebalanced
    \graph [mytree] {
      root [x = 7.2cm,y=-2.5cm,draw=none,as={\phantom{\(2~~~6~~~9\)}}];
      root --[gray] { a [as={\color{gray}\(T_1\)},subtree]};
      root --
      { es [as={\(4\)}] --
        { e [as={\(T_2\)},subtree],
          b [as={\(T_3\)},subtree]},
       7 [as={\(7\)}] --
        { c [as={\(T_4\)},subtree],
          d [as={\(T_5\)},subtree] }};
      root --[gray]
      { f [as={\color{gray}\(T_6\)},subtree]}};
    \node[anchor=west, treenode,draw=gray] at (root.west) {\color{gray}\(2~~~6\)};
    \node[anchor=east, treenode,draw=gray,fill=none] at (root.east) {\(6~~~\color{gray}9\)};

    \draw[algs4arrow] (4.cm, -3.6cm) -- +(.8cm, .3cm);
    \draw[algs4arrow] (5cm, -1.6cm) -- +(.8cm, -.3cm);

  \end{scope}

  %% Merge
  \begin{scope}[shift={($(left) + (10.6cm,-11.4cm)$)},
    every node/.style={anchor=south},
    note/.style={font=\footnotesize},
    subtree/.style={draw=none,fill=none},
    edgenote/.style={algs4red,->,>=stealth',shorten >=3pt}]

    % Background & lead
    \begin{scope}[on background layer]
      \fill[black!15] (-0.07, 0.4) rectangle (8, -6);
    \end{scope}
    \node[lead,draw=none,shape=rectangle] at (0, 0.3) {Merging.};

    %% Left heavy
    \graph [mytree] {
      root [x = 2.5cm,y=-1cm,draw=none,as={\phantom{\(2~~~7~~~9\)}}];
      root --[gray] { a [as={\color{gray}\(T_1\)},subtree]};
      root --
      { 46 [as={\(4\)}] --
        { b [as={\(T_2\)},subtree],
          c [as={\(T_3\)},subtree]},
        es [as={},minimum size=2pt,fill=black] --
        { e [as={\(T_4\)},subtree]}};
      root --[gray]
      { f [as={\color{gray}\(T_5\)},subtree]}};
    \node[anchor=west, treenode,draw=gray] at (root.west) {\color{gray}\(2~~~7\)};
    \node[anchor=east, treenode,draw=gray,fill=none] at (root.east) {\(7~~~\color{gray}9\)};

    %% Right heavy
    \graph [mytree] {
      root [x = 2.5cm,y=-4cm,draw=none,as={\phantom{\(2~~~4~~~9\)}}];
      root --[gray] { a [as={\color{gray}\(T_1\)},subtree]};
      root --
      { es [as={},minimum size=2pt,fill=black] --
        { e [as={\(T_2\)},subtree]},
       67 [as={\(7\)}] --
        { b [as={\(T_3\)},subtree],
          c [as={\(T_4\)},subtree]}};
      root --[gray]
      { f [as={\color{gray}\(T_5\)},subtree]}};
    \node[anchor=west, treenode,draw=gray] at (root.west) {\color{gray}\(2~~~4\)};
    \node[anchor=east, treenode,draw=gray,fill=none] at (root.east) {\(4~~~\color{gray}9\)};

    %% Rebalanced
    \graph [mytree] {
      root [x = 6.5cm,y=-2.5cm,draw=gray,as={\color{gray}{\(2~~9\)}}];
      root --[gray] { a [as={\color{gray}\(T_1\)},subtree]};
      root --
      { es [as={\(4~~7\)}] --
        { e [as={\(T_2\)},subtree],
          b [as={\(T_3\)},subtree],
          c [as={\(T_4\)},subtree] }};
      root --[gray]
      { f [as={\color{gray}\(T_5\)},subtree]}};

    \node[font=\itshape\footnotesize, text width=1.7cm,align=center] (may) at
    ($(root.north)+(0cm,1cm)$) {May become a \texttt{1-}node};
    \draw[edgenote] (may.south) -- (root.north);

    \draw[algs4arrow] (3.8cm, -3.6cm) -- +(.8cm, .3cm);
    \draw[algs4arrow] (4.7cm, -1.6cm) -- +(.8cm, -.3cm);

  \end{scope}

  %% Box
  \draw [algs4red,line width=3pt] [use path=\theframe];

\end{tikzpicture}
\end{document}
